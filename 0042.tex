% \subsection{§042 アイテムの確認 11/22 (thu)}
\subsection{§042 アイテムの確認 11/22 (thu)}

% 「先輩、めっちゃ面白かったです!」\\


% ドリーに帰ると、三好が目をきらきらさせて出迎えてくれた。


% モニタ越しに見れば、結構なアクションムービーだろうが、俺はもうシャワー浴びてメシ喰って寝たい……\\


% 「先輩、ところでヒールポーションランク5の値段って知ってます?」


% 「あー、よく知らん」\\


% でしょうね、と三好が面白そうに笑って、冷たい水を渡してくれた。\\


% 「ヒールポーションのランク1はJPYだと100万~200万くらいです」


% 「へー」\\


% 安くはないが、プロの探索者にとって、特にべらぼうに高価というわけではなさそうだ。


% ごくりと水を一口飲む。よく冷えていて、体に染み渡る感じがする。思ったよりも疲れているのかな。\\


% 三好の説明によると、ポーションのランクと効能は以下のような感じらしい。\\


% ランク1は大体、単純骨折が修復される程度らしい。いわゆるテニスエルボーや野球肘くらいまではきれいに治る。腱の断裂も修復するらしい。


% ランク2になると、複雑骨折や眼球の損傷、割かれたお腹などをきれいに修復する。


% ランク3は、広範囲にわたる火傷や、きれいに切断された体がくっつく。


% ランク4は、ぐしゃぐしゃに潰された手足が元に戻る。


% ランク5は、半分欠損していても元に戻る。


% ランク6は、9割欠損が元に戻る


% ランク7は、失われていても48時間以内なら元に戻る


% ランク8-10は未発見。\\


% そうすると、アーシャの場合は8以上のランクが必要だったわけで、見つかってないんじゃ買えるはずがないよな。\\


% 「それぞれにランクの価格は、ランク1の相場を基準にして、後は大体ドロップの稀少性によって計算されています。具体的には、大体、ひとつ前のランクの価格xそのランクです」


% 「つまり、ランク1が100万なら、ランク2は100万x2で200万。ランク3は、200万x3で600万ってことか?」


% 「そうです。もっとも高ランク品は数が少ないので実際の取引価格はバラバラですけどね」\\


% つまりは、ランク1の価格x当該ランクの階乗かよ……\\


% 「それってランク1が100万だとしたら、5は……」


% 「1億2000万ですね」


% 「おう……」\\


% そりゃ、あの姉弟が驚くのも無理はない。\\


% 「大体ランク5なんて、一般にはなかなか出回りませ――」\\


% おれは静かに今回手に入れたアイテムをテーブルの上に取り出した。\\


%   --------


% ヒールポーション(5)


% キュアポーション(7)


% 牙:ヘルハウンド×8


% 皮:ヘルハウンド×3


% 舌:ヘルハウンド


% 魔結晶:ヘルハウンド×8


% 皮:ハウンドオブヘカテ


% 角:ハウンドオブヘカテ×3


% 魔結晶:ハウンドオブヘカテ


%   --------\\


% 「――先輩、まさかこれ?」


% 「今さっきの成果物」


% 「ドロップ率がおかしい気がするんですけど……」\\


% そう言って三好は、ごそごそとアイテムに触れながら、内容を確認していた。\\


% 「LUCのせいかもな」\\


% 淡い黄緑色をしたポーションらしきものに触れたとき、三好は思わず顔を上げた。\\


% 「先輩! こ、これ。このキュアポーション、ランク7ですよ?!」\\


% 俺にはその価値がいまいち実感できないが、さっきの話がそのまま適用されるなら、7の階乗は5040だ。


% まあ驚くと言えば、驚くか。\\


% 「なに落ち着いてるんですか。キュアポーションは病気の治療に使われるポーションですが、ランク7だと大抵の不治の病は全快するんですよ?」


% 「なんだって?」


% 「白血病はおろか、認知症を完治した報告例があります」\\


% 俺は水を吹き出しかけた。認知症って病気扱いなのか?\\


% 「神経細胞の回復なんて、どちらかというとヒールポーションの領域じゃないのか?」\\


% 第一失われた記憶ってどうなるんだよ?


% ハードウェアが元に戻っただけでどうにかなるもんなのか?\\


% 「そうなんですよね。だから、その辺も先輩が言ってた『意識』の問題なのかも知れません」


% 「それって、認知症が神経細胞の『ケガ』だっていう認識を持てば、ランク6くらいで全快しちゃうかも知れないってことか?」


% 「ありそうですねぇ。その実験をやれるところは少ないというか、ほぼないと思いますが」\\


% なにしろ高ランクポーションは、それを欲する人の列で埋まっている。


% どうなるかわからない実験に使えるような状況ではないらしい。\\


% 「ランク4までのキュアポーションは比較的安価に流通しています。4でも難病のうちいくつかは完治するようです」\\


% 全体の出現バランスはヒールポーションとそれほど違いはないが、必要になるのは主に一般人で、エクスプローラに大きな需要のあるヒールポーションとは需要構造が異なっているため、少し安いそうだ。\\


% 「安価?」


% 「ランク4で、最低1920万ですね」


% 「それの何処が安価なんだよ?」\\


% 俺は呆れながらそう尋ねた。


% ヒールポーションと同じ計算式なら、ランク1が80万くらいだ。\\


% 「実際に難治の治療に使われている金額。つまり保険から支払われている金額に比べたら遥かに安価な場合が多いんです。厚生労働省が保険支出の圧縮に、分配組織の設立を検討しているくらいです」\\


% ものすごく治療に金がかかる病気を、さっさとポーションで直して支出を圧縮するってことか。そういや、白血病のキムリアなんかアメリカなら成功報酬とはいえ47万5千ドル。日本だと治っても治らなくても3300万ちょっとだもんな。2000万なら安いのか……\\


% 健康保険の高額療養費制度を利用すれば、患者が払うのは自己負担限度額だけ。限度額適用認定証を利用すれば、一時支払いも不要だ。win-winの関係と言えばその通りだが……\\


% 「それ、薬を開発している企業にとっちゃ、ガンなんじゃないの?」


% 「まだ流通量が少ないですから問題視されていませんが、流通量が増えたりしたら死活問題ですからね。そんな組織は作らせない可能性もありますね」\\


% ポーションを大量に流通させたりしたら、命を狙われかねないってことか。\\


% 「現在完治が極めて難しい、または、不治と呼ばれる病気は、ランク5以上から効き始めるんですが、ドロップするモンスターの難易度が跳ね上がるため、めったに流通しません」


% 「もっとも、このクラスの病になると、薬を開発しても患者が少なすぎて利益が出ませんから、人類に貢献するという意味では素晴らしいものがありますけど、出資者はいないでしょうね」


% 「つまりランク5より先は、流通したとしても、それほど薬の開発に影響を与えないってことか?」


% 「おそらく。で、現在のランク7キュアポーションの価値ですが――ざっと40億3200万ですね」\\


% 流通しないので、確率から計算しただけの価格ですが、と付け足された。とはいえ――\\


% 「誰が買うんだよ、そんな薬」


% 「遺産配分を決めないまま、急性で認知症になっちゃった大富豪とかですかね」


% 「なるほど……って、じゃあランク7ヒールポーションも?」


% 「50億ちょっとくらいです」


% 「マジデスカ」


% 「エクスプローラは財産ですからね。死んでしまえば与えたスキルオーブもパー。各国のトップエクスプローラは絶対に死なせてはいけない国有財産とみなされていると思いますよ」\\


% それでも危険なダンジョン内に送り込まなければいけないんだもんなぁ。


% 沈むのが前提の軍艦に、保険は掛けられないか。\\


% 「取り替えのきかない機材を保護するのに、50億くらい安いって?」


% 「メンテナンス込みなら、戦闘機1機分以下ですからね」


% 「ゆがんでる気がする」


% 「実際、取得はそれくらい大変でコストもかかるみたいですよ。本来こんなにほいほいドロップするようなものでは……」\\


% エクスペディションスタイルで、なんども潜ってゲットするわけだしなぁ。たしかにそうなのかもしれない。\\


% 「じゃあ俺が――」


% 「先輩が社会正義に目覚めるのは勝手ですが、一人で頑張ったところで、中間業者がぼろ儲けするだけです。決して価格は下がりませんよ。供給が全然足りないんですから」


% 「――だよな」\\


% アイテムはオーブとは違って、時間制限が緩い。


% つまりはそこに中間業者が暗躍する――じゃなくて商売するチャンスが生まれるわけだ。\\


% 「言っておきますが、手に入れられない人達に配って歩くのもお薦めしません」


% 「なんで?」


% 「あいつは貰えたのに、俺は貰えない。それが我慢できない『俺』が世の中には大勢いるからですよ」\\


% 全員に配るのは無理だ。


% だからそこには命の選別がある。恨まれるのは当然か。\\


% 「先輩も見たでしょ、さっきの男の子。ああいうのが……まあ普通とはいいませんが、沢山いると思って下さい」


% 「はあ。ままならないねぇ」


% 「消耗する美術品や宝石だと思えば腹も立ちません」\\


% さすが三好、割り切ってらっしゃる。\\


% 「で、牙や皮や角はともかく、舌とか魔結晶ってなんに使うんだ?」


% 「魔結晶は、超高エネルギーの結晶らしいですよ」


% 「なんだそれ。意味わからん。石油の代わりにでもするのかよ?」


% 「それなり以上のモンスターからは、時々産出するアイテムで、一般にはあまり知られていませんが、化石燃料の少ない国は超注目しているそうです。付いた名前がクリーンなプルトニウム」


% 「そんなにか?!」\\


% ダンジョンが出現して以来、世界は日単位で変貌を遂げている。


% そのうち人類は石油のようにダンジョンに依存して生きるようになるかも知れないな。というか、なりかかってるのか。\\


% 「実用化には、いろいろと壁があるそうですが……中でもこれは、今のところの最高品質のひとつでしょうね」\\


% 怪しく輝くハウンドオブヘカテの魔結晶は、サイズもそれなりに大きく、ソフトボールくらいあった。


% ヘルハウンドの方は、2cmもない。\\


% 「で、舌は?」


% 「わかりません。そうとうなレア素材だとは思いますが――あ、ヒットしました。って、ええ?」


% 「どうした?」


% 「あの……食材だそうです」


% 「魔物を喰うのかよ!」\\


% タンパク質の構造だとか、遺伝子の問題だとか、言いたいことは山ほどあるが、地球上にいない生物の体を喰うってどうなの?


% 遺伝子組み換え食品の危険性がどうとか言ってるのとは、レベルが違うんじゃないの??\\


% 「検疫とか、食の安全性とかどうなってんだ?」


% 「ダンジョンの防疫と同じで意味が分かりません。ただ……」


% 「ただ?」


% 「美味しいそうですよ。じゅるり」


% 「いや、あのな……」\\


% 確かにうまいは正義かもしれん。


% ある日突然人間じゃなくなる危険があるのかもしれなかったとしても。\\


% 「あとですね」


% 「ん?」


% 「ダンジョン産の食品を食べると、様々な能力が向上する傾向があるそうです」


% 「……なんだと?」\\


% また、ステータス関係か?


% 何かに手を出すことで、能力に差ができるのなら、人は競争のために同じものに手を出さざるを得ない。相手国が核兵器を持てば、自国も持たねばならないのと同じ理屈だ。


% ステータスの上昇は確かに人類を進歩させるかも知れないが……それってダンジョンへの依存はそのまま深まることになる。\\


% 「いきなりできたダンジョンが、いきなりなくなったりしない、なんて、どうして思えるんだと思う?」


% 「空が落ちてくる心配をするよりも、今の利益をむさぼるほうが、企業として重要で、かつ健全だからですよ」


% 「空が落ちてくることを防ぐ何かを作っても、落ちてくるかどうかもわからなきゃ、買い手がいないってわけか」


% 「そういうことです」\\


% はぁー、とは思うが、所詮我々一般人が考えても詮ないことだな。\\


% 「ま、その辺は偉い人に任せておくしかないか。で、念願の異界言語理解を手に入れたわけだが……これってどうする?」


% 「どうするって、売るしかないでしょう。使いますか?」\\


% 「世界の命運を分ける鍵になるなんて、絶対にイヤだね」


% 「ですよね。まあ、1個は売りに出してみたい気もしますが……USとRUがどこまでも競り合いそうな気がしません?」


% 「それなんだけどな。高額なのは今のうちだけだぞ」\\


% 俺は、イレギュラーなレアモンスターが持っていたオーブのドロップ率のことを話した。\\


% 「確かに先輩の言うとおり、碑文を読んでくれと言わんばかりのドロップ率ですね……」\\


% 三好は首をかしげながらそう言った。\\


% まあ、問題は「誰が」そう考えているのかってことだな。\\


% 奇妙に地球文化にマッチするダンジョン内のルール、荒唐無稽に思えるパッセージ説の肯定、依存性のある麻薬のように広がるダンジョンの影響。


% 実は地球人全員が生まれたときから管理されてて、この世は全部仮想空間でマトリックス的なものだと言われても、今なら納得しちゃいそうだよ。そのうちスミスに襲われるに違いない。\\


% さしずめメイキングを得た俺は、大統領の暴走を止めるために暗殺を決意するクリストファー=ウォーケンの役所ってか?(*1)\\


% 「いずれにしても勝手にオークションに出すのは、あまりに義理を欠く気がしますし。あ、アイテムは一応先輩が持っていてください。時間経過の影響がわかりませんから、念のため」


% 「わかった。じゃ、俺はシャワー浴びてメシ喰って寝る」\\


% 「了解です。私は今日のデータを整理してから寝ます」\\


% そこにあるのは、今回のダンジョン行で録画した動画データや、作成された3Dマップデータや、モンスターの経験値を初めとするパラメータ群だ。\\


% 「これって、実は世界中の研究者がよだれを垂らしまくるくらい、価値のある情報なんですよね……」


% 「盗まれることを気にするんなら、PCごと収納しておけよ。使うときだけ出せばいいだろ」


% 「なるほど! でもずっとデッパな気も……」


% 「ずっと徹夜するのかよ、まったく。じゃあ、事務所にSECOMでも入れるか?」


% 「もし何かが来るとして、民間の警備会社でどうにかなるようなところが来ますかね? 一応レーザー盗聴対策なんかは施してありますけど」\\


% 事務所の改装費用がやたらとかかってたのはそれか。


% 本来は目立たないのが一番なんだが……最近ちょっと、それは無理かもなぁと思い始めた。なんというか、乗ったトロッコにブレーキがないことを、坂道を下り始めてから気付いた感じだ。\\


% 「とにかく重要なものはワンパッケージにまとめて、いざとなったら収納して逃げろ」


% 「ですね。セーフルームでも作りますかねぇ。先輩が助けに来てくれる時間を稼げるような」


% 「俺は、どっかの国と闘うのは嫌だぞ」


% 「え、助けに来てくれないんですか?」


% 「う……いや、行きそうな気がする」\\


% そこが先輩のダメなところで、素敵なところですよ、と三好が笑った。\\

