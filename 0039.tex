% \subsection{§039 探索の始まり 11/22 (thu)}
\subsection{§039 探索の始まり 11/22 (thu)}

% 「それじゃ、三好、準備はいいか?」


% 「大丈夫ですよー。先輩こそ、ちゃんとご飯は持ってるんでしょうね?」


% 「大丈夫だ。たっぷり仕入れた。じゃ、いくか」


% 「はい」\\


% 初心者装備とはいえ、初めてのまともな装備だ。なんとなく身が引き締まる思いがする。


% 俺達は、それなりにやる気に満ちながら、代々木の地下へと降りていった。\\


% 手ぶらだと目立つので、とりあえずLBAのミニシールド装備して、ブローニングのトマホークを腰に下げておいた。\\


% いつもとは違う人の流れに乗って、2層への階段を目指す。


% 最大の目的地は、14層のムーンクランだが、各階のモンスターの経験値測定も重要な作業だ。\\


% 各階の過疎地域や環境は、三好と一緒に綿密に予想した。


% メットにセットしたアクションカメラや深度センサーも常時作動しているらしい。


% 深度センサーは自動でダンジョンの3Dマップを作成するものだそうだ。JDAが提供しているダンジョンビューのオリジナル版らしい。どうせ行ったことのある場所をマッピングするんだからそのついでらしい。\\


% バッテリーは三好が山ほど買い込んでいた。情報には計り知れない価値がありますからね、とのことだ。それは正論とは言え、保管庫や収納庫がなければ、ほとんど不可能な言ってみれば戯言だ。その点、俺達にその心配はなかった。\\


% 「初心者さんですか?」


% 「あ、はい」\\


% 3人組のパーティの男が、俺たちに話しかけてきた。装備を見てそう思ったのだろう。\\


% 異世界ものやVRMMOものなら、ここで強盗を疑うところだが、代々木でその手の事件はほとんど発生していない。


% 日本人の気質もあるだろうが、ダンジョン利用者は完全に個人情報をJDAに把握されているから、匿名犯罪は起こしにくかった。


% ここで初心者を襲って何かを得たところで、現代日本じゃほとんど割に合わない。リターンに比べて、リスクが大きすぎるのだ。\\


% 吉田と名乗ったその男は、2層へ向かう道中で、いろいろと初心者の狩り場などについて教えてくれた。\\


% 「あと、その防具だと5層より先には降りない方が良いですよ」\\


% 5層から現れ始めるボア系と呼ばれるモンスター群の突進が、初心者装備では受け止めきれないそうだ。\\


% 5層はアマチュアのファン層とプロ層を分けているフロアだ。\\


% 代々木では4層までのモンスターは、所謂「通常アイテム」と呼ばれる、モンスターを倒すと高確率で得られるアイテムをドロップしない。


% つまり、代々木ダンジョンで生計を立てるものは、必ず5層以降へ降りる必要があった。


% 逆に言うと、プロ層とアマチュア層は、その縛りで分離されているため、探索への姿勢の違いで起きるいざこざも抑制されていた。\\


% 「わかりました。ありがとうございます」\\


% 丁度2層へ降りる螺旋階段で、お礼を言って別れた。


% そうして俺たちは、初めて2層へと降りたった。\\


% 「知ってはいましたけど、なんだか不思議な光景ですよね……」\\


% ダンジョンに、空があったっていいじゃないか。


% つい昔の芸術家の台詞が頭をかすめる。もっとも、顔と違って、どんなものにも空はないが。あ、中身をなくせば空(から)は作れるか。\\


% そこにはダンジョンの中とは思えない空と、草原や丘や森があった。この傾向は9層まで続くらしい。


% 振り返ると、俺達が出てきた階段の入り口は、少し急な壁を持つ丘の中腹に口を開けていた。中を覗いても緩やかなカーブを描く上り階段の先は見えない。普通なら丘の天辺を突き抜けているだろうが、丘の上に天に昇る螺旋階段は無かった。\\


% 「確かに不思議だ」\\


% 気を取り直して前を向いた。\\


% 2層に棲息しているのは、ゴブリン・コボルト・オークなどの定番人型系モンスターに、ウルフなどの獣系が主体だ。\\


% 「とりあえず、2層のゴブリンとコボルトか」\\


% ふたりで作ったマップを確認しながら、俺たちは過疎エリアに向かって歩き出した。\\


% 2層の過疎エリアは、単に3層へと降りる階段に向かうルートとは逆方向だというだけだ。


% 1層ほどではないが、2層も下層に降りる階段が比較的近距離にあるため、逆方向は過疎率が高い。


% 当然奥地へ行けば行くほど過疎っている。\\


% 俺は人のいない場所まで移動すると、全ステータスを本来の値に戻して、走ってみた。\\


% 「う、うぉっ!」


% 「ちょっ! 先輩! 何処行くんですか! 待って下さいよ!」\\


% 体が羽根のように軽く凄い速度で移動できるが、知覚自体は、まるで時間を引き延ばしたようで、体の制御は容易だった。\\


% 「凄いな、ステータス」\\


% 代々木ダンジョンの既知のエリアに罠は確認されていない。というか、世界中のダンジョンで罠が見つかった例はない。


% 理由はわからないが、そのあたりがフィクションと少し違うところだ。\\


% もっとも、罠があるほうが謎だと思う。一体誰が仕掛けてるって言うんだよ。


% 普通の屋敷を調べていたら、絵の裏にボタンがあって、押すと別の部屋で壁が開いて銃弾が見つかる、なんて、リアルだったら作ったやつは頭がおかしいとしか思えない。\\


% そういうわけで、環境セクションの構造にもよるが、結構な早さで移動してもそれほど問題にはならなかった。


% 角を曲がったときモンスターとぶつかって恋に落ちるかも知れない可能性が多少高くなる程度だ。\\


% もっとも、三好がついて来るのは無理だろう。\\


% 「先輩。そんなに高速で移動するなら、私を担いでいって下さいよ」\\


% いや、緊急でもないのに無理だから。まあ、普通に歩こうか。


% 数分進んだところで、まっすぐな通路の奧に、小さな人影のようなものを見つけて立ち止まった。\\


% 「第一ゴブリン発見?」


% 「ですね。まずは経験値の確認をしましょう」


% 「了解」\\


% 俺はメイキングを展開すると、鉄球を取り出した。\\


% アイテムボックスから出すと同時に加速させるワザだが、俺には出来なかった。\\


% 三好は自在に使いこなしていたから、収納庫ならできて保管庫にはできないのかもしれない。


% 取り出す際に内部で加速させているんだとしたら、保管庫は時間の経過が0だから、加速させることなど出来るはずがない。\\


% もちろん俺と三好の才能の差だという可能性はある。もしそうなら、ちょっと泣ける。


% というわけで、俺は6cmの鉄球を取り出すと、ただそれを投げつけた。\\


% 指からボールが離れた瞬間、パンという音と共に、ゴブリンらしきものの頭がなくなっていた。\\


% 「は?」


% 「おおー」\\


% 俺は一瞬唖然としたが、これが、STR/AGI/DEX ALL 100 のパワーなのだろう。考えてみれば31層をクリアした男達よりも高いステータスなのだ。おそらく、ではあるが。\\


% ゴブリンの経験値は0.03だった。スライムよりちょっとだけ多い。\\


% 次に遭遇したゴブリンには、ウォーターランスと名付けられた水魔法を使ってみた。鉄球には限りがあるし、森エリアでは回収が面倒だったからだ。\\


% 魔法のスキルオーブにはローマ数字の付いたものと、無印がある。


% 数字の付いたものは、ナンバーズと呼ばれていて、その番号に対応した魔法が最初から使える反面、他の魔法を覚えない。


% 無印は経験や訓練などよって、ナンバーズと同等の魔法やオリジナルの魔法を身につけることが出来るが、修得の難易度が高いということだ。\\


% スライムから出たオーブは無印だった。


% 俺たちはすでに知られているナンバーズの魔法を参考に、クリエイトウォーターとウォーターランスっぽい魔法を身につけていた。\\


% この水魔法は、初期状態で、1本の槍を作るのにMPを1消費した。


% 効果は鉄球ほどではなかったが、一撃でゴブリンを葬り去ることに変わりはなかった。\\


% 「こりゃ楽でいいや」\\


% 以降、俺はウォーターランスを使いまくった。火魔法と違って、森にダメージを与える可能性がないのがいい。


% ラノベの世界なら使い続けていると効果が強化されたりするのが定番だが、ゲームの世界では使い続けても大抵効果は変わらない。


% 現実がどちら寄りなのかという興味もあったのだ。\\


% ステータスはフルセットしてあるので、現在の最大MPは190だ。メイキングで確認したところによるとMPは1時間でINTと同じ値が回復するようだった。


% もっともこれが普通なのか、超回復の力なのかはわからない。細かいことは三好が考えるだろう。


% 俺はひたすら数値を記録した。案の定2匹目のゴブリンのSPは、0.015だった。\\


% ゴブリンたちは事前の調査の通り、まとまって棲息していた。\\


% しかも過疎エリアだけに、間引かれる頻度も低いらしく、2時間もたつころには100匹が近づいてきていた。


% 途中、ウルフを何匹か倒している。こちらもウォーターランスで瞬殺だった。さすがはINT100だ。\\


% 最初のウルフの経験値は0.03。ゴブリンと同じだ。すでにゴブリンの経験値は、0.003になっていたので、件の匹数による経験値減少問題は種類別に計算されると言うことだろう。


% そうして、91匹目のゴブリンを倒したとき、それは起こった。\\


% 「え?」


% 「どうしたんですか?」\\


%   --------


%   スキルオーブ DEXxHP+1 1/    5,000,000


%   スキルオーブ   早産 1/   10,000,000


%   スキルオーブ   早熟 1/  800,000,000


%   スキルオーブ   促成 1/1,200,000,000


%   スキルオーブ   早世 1/2,000,000,000


%   --------\\


% ダンジョンに潜る前に、モンスターを倒した数の下ふた桁は、ぴったり00にセットしておいたはずだ。


% 因みにウルフは9匹を倒していた。\\


% 「1種類100匹目じゃなくて、モンスターを倒した数100匹目で発動するのか……」\\


% このことは非常に重要だ。


% なんといってもムーンクランシャーマンを100匹倒さなくて良いというだけで、朗報なのは間違いない。\\


% 「それって、うまく調整すれば、ボスのオーブも取り放題じゃないですか!」\\


% 目が¥マークになった三好が、白鳥の湖を踊っている。そりゃそうかもしれないが、まずはボスを倒せないとだめだろ。


% それはともかく、このオーブ、なにかこうヤバそうなのが並んでいる。ていうか「早世」ってなんだよ! 年寄りが使ったらどうなるんだよ?!\\


% 俺はオーブを読み上げて、三好に伝えた。\\


% 「促成と早世は未登録スキルですね」\\


% どうやら、スタンドアローンのスキルデータベースを持ち込んでいるようだ。\\


% ゴブリンは大抵の人が倒すモンスターだ。


% カード所有者が1億人くらい居るわけだから、ひとりが12匹たおせば、確率的には促成が1つ手に入りそうなものだけど……\\


% 「カードを手に入れるためだけに倒した人が相当数いて、後はすぐにゴブリンを卒業しちゃうんじゃないですか。売れる素材がないですから」


% 「だよなぁ」\\


% 俺だって、ゴブリンを永遠に狩り続けるのは嫌だもんな。経験値はほとんどスライムと変わらない上に、ドロップアイテムも聞いたことがない。\\


% 「ゴブリンは、ドロップアイテムじゃなくて、巣に集めてあるアイテムを狙うのが主流らしいですよ」


% 「え、なにそれ?」\\


% 今までも巣らしいところは結構あったが、そんなもの、探したことがなかった。\\


% 「ドロップアイテムと違って、探さないと見つかりませんからね。GTBって呼ばれてるそうです」


% 「Goblin's Treasure Box かよ。全くしらなかった。早く言えよ」


% 「探すのに時間が掛かる割に、それほど大したものは入っていないそうですから。最高でランク1のポーションだそうです。さっさと進んだほうがいいかなと思いまして」


% 「それでもちょっとした宝探しだな」


% 「ですね」\\


% デートで宝箱探しってのもいいかもな。\\


% 「ゴブリンを倒して歩くデートはお薦めしませんよ。普通の女性なら引かれます」


% 「何故分かった!」


% 「先輩は、女の子のことを考えると、鼻の穴が膨らむんですよ」


% 「マジで?!」\\


% どうでしょう? と三好には躱された。本当だったらいかんな。\\


% DEXxHP+1 はいわゆるハズレスキルだが、以前三好と話した感じでは、将来的に重要になりそうなものだ。


% しかもドロップ確率が五百万分の一だ。世界中で結構な数がドロップしているだろう。\\


% 「早産は、子供を早く生むスキルですね。あまりの名称に最初は豚に使われたそうです」\\


% 人間以外でもモンスターを倒しさえすれば、カードが現れるそうだ。ただし、野生の状態ではカードも紛失してしまうだろう。


% なお、名前がないからか、名前もランキングも表示されないらしい。名前の付いたペットに使うとどうなるのかは気になるところだ。\\


% カードを取得した動物は、当然のようにオーブを使えるようになるそうだ。どうやってオーブを使用するのかはわからないが、豚の場合は食べさせたらしい。


% 使用されたことはオーブの使用効果で目に見えるし、カードがあればカードにも記載される。動物の場合もそれで取得を確認できるようだった。\\


% 早産を使われた豚は、妊娠後、わずか12日で出産した。そして、このスキルの凄みは、その子供が全て正常な子豚だったことだ。


% つまりこのスキルは、妊娠期間を約1/10にするスキルだったのだ。ただし親の方はそれなりに衰弱するらしい。通常と比べて、時間あたり10倍のエネルギーを使うからだと予想されている。


% 現在では、動物を使った、スキルの遺伝を調べる実験に、活用されているらしかった。\\


% 「早熟は、今までに2つしか見つかっていないそうです。どうやらダンジョンでの成長速度が非常に速くなるらしいですけど……」


% 「名前だけ見たら、すぐに頭打ちになりそうだよな」\\


% 十で神童十五で才子二十過ぎればただの人ってやつだ。\\


% 「しかし、人間には使い辛いオーブばっかりだな、これ」\\


% 結局安心して使えそうなのは、xHP+1しかないありさまだ。\\


% 「まあ、とりあえずはレアリティで押さえておくか。しかし、早世はなぁ……暗殺アイテムかっての」\\


% 早世――取得確率は20億分の1――は、どう見ても寿命を縮めるスキルだとしか思えない。早産を参考にするなら10分の1か?


% しかしそれだとデメリットしかないわけで、暗殺に使うにしても時間がかかりすぎるだろう。


% たぶん何かそれに見合うメリットが――天才になるとか?――あるのかもしれないが、試す方法がなかった。もちろん自分で使うなんてお断りだ。\\


% 「ここは促成にしておくか」\\


% そうして俺は、「促成」を手に入れた。\\


% ◇◇◇◇◇◇◇◇\\


% ゴブリンはまだまだ尽きないし、MPも1時間で100は回復するから、もう2時間程狩り続けた。


% 充分データを取ったらしい三好も、ちまちまとウォーターランスで倒している。加速鉄球なら一撃だが、いかんせん玉を消費するからだ。\\


% 「鉄球と違って、いまいち快感が……」\\


% なんて言う、ちょっと危ない人になっているぞ。\\


% そうして、あらかじめ見つけておいたコボルトを、丁度100匹目になるタイミングで倒した。\\


%   --------


%   スキルオーブ AGIxHP+1 1/     20,000,000


%   スキルオーブ    AGI+1 1/     50,000,000


%   スキルオーブ 生命探知 1/  1,200,000,000


%   スキルオーブ 交換錬金 1/ 16,000,000,000


%   --------\\


% 「交換錬金以外は知られていますね」\\


% 残りのスキルは既知のようだ。生命探知はここでは激レアだが、主にウルフ系列の上位種がドロップするらしい。\\


% 「コボルトは、コバルトの語源にもなっていますから、錬金っていうのもわからないでもないですけど……交換ってなんでしょうね?」


% 「何か対価を要求されそうだよな……実に寓話っぽく」\\


% 160億分の1は確かにレアだが……\\


% 「しかし、コボルトやゴブリンのオーブは、どれも罠感満載だな」


% 「悪戯好きな妖精の性(さが)ですかね?」\\


% 危険すぎて売ることもできそうにないし、試すのも嫌だ。鑑定を手に入れたらまたこよう。


% 俺たちは大人しく、生命探知を手に入れた。\\


% 「とりあえず、取得できるオーブも分かったし、先へ進むか?」


% 「そうですね。ウルフとかは下でも出ますし、代々木の2~4層は少しずつ相手が強くなったり出現頻度が変わるだけで、種類はほぼ同じですからとりあえず5層を目指しましょう」


% 「よし、いくか」\\


% 「先輩、その前に誰もいないこのへんでご飯にしましょうよー。お腹減りましたー」


% 「まあそうか。じゃ、出すか? あれ」


% 「ふっふっふー、移動拠点車ドリーちゃんのデビューですよ!」\\


% 三好は少し広い場所で、拠点車を取り出した。


% 俺たちはそれに乗り込むと、デパ地下総菜とお弁当を取り出して、お昼ご飯にしたのだった。\\

