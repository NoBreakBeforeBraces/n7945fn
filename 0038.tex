% \subsection{§038 準備 11/21 (wed)}
\subsection{§038 準備 11/21 (wed)}

% 「先輩。キャンピングカーが納車されるそうです。結局チタンの板はビルダーに付けて貰いました」


% 「当面はそれで凌ぐとして、キャンピングカーのビルダーでもいいんだけどさ、1LDKくらいの広さのダンジョンハウスをどっかが作ってくれないかな」


% 「過酷なダンジョンで、キャンピングカーじゃ、いつまでも持ちそうにないですもんね。完全な密封や循環系が必要なら宇宙関係企業でしょうか」


% 「そこまでは期待していないけど……ダンジョン内で普通のテント暮らしとか俺達には無理そうだしなぁ……」


% 「現代の便利さに浸りきった、軟弱な我々には無理だと思います」


% 「待て。軟弱だからじゃないぞ? 二人だからだ。見張りが二人組なら寝られないだろ?」


% 「まあ、そう言うことにしておきます。でも、そう考えると、軍の人って凄いですね」\\


% まったくだ。


% 日本の自衛隊は、エクスペディションスタイルで代々木へ潜っていったらしいが、サイモン達は、純然たるアドベンチャラースタイルでアタックしているらしい。


% いろいろ軍のハイテクアイテムがあるんだろうが、それでも基本は見張り&探索者用寝袋だろう。すげぇぜ。\\


% 「一応途中まではトレーラーで運んで貰って、途中からうちの前庭まで自走で納車して貰えるそうです」


% 「公道走れるんだ?」


% 「フロントの板は後付けで、納車後に付けてくれるそうですから。でもそれ以降は無理だと思います」\\


% 車検はともかく前が見えないんじゃ運転はできないだろうな。


% 拠点として使うだけだから、別に困らないが。\\


% 「ところで、武器や防具はどうします? 2F以降は、さすがにないと拙くないですか?」


% 「ないと目立ちそうだもんな。親切な探索者がいちいち意見してくるのも、数が増えると面倒だし」\\


% 「どうせ高価な防具でも完全には守りきれないんだから、超回復とVITを信じて動きやすさ重視で。御劔さん達が最初に使ってた、初心者装備でいいよ」


% 「私たち、ランクもGですから、妥当ですよね。で、武器は? ここは格好良く聖剣シリーズとか使います?」\\


% 三好がネットで武器を検索しながら笑っている。\\


% 「なんだそれ」


% 「あるんですよ、ゲーム会社が武器メーカーとタイアップして作ったようなのが」


% 「どう聞いてもネタ装備だろ、それ。それに剣なんて使えないって」


% 「近づくのがイヤなら、やっぱり飛び道具ですかね?」


% 「あ、相手の飛び道具を避けるのに盾は欲しいな」


% 「チタンのフライパンじゃだめですか」


% 「さすがにちょっとな」\\


% 確かにあれは、スペックだけを見るなら高CPだが、保護面積が狭いからな。\\


% 「さすがにそう言ったものは、ダンジョン施設のショップじゃないと買えないですけど……」\\


% 三好がすばやくPCを操作して、よさそうな盾を絞り込んだ。\\


% 「ブンカーシールドとかありましたけど、180kgあります」


% 「それはさすがにいらん」


% 「USのSWATで使われてる、プロテック・タクティカルのパーソナル・バリスティック・シールドが10Kg弱、もっと小さくて良ければ、LBA社のL10-Sミニシールドとかあります。アラミド繊維で作られてて、3.2Kgだそうです。体は全部隠れませんけど」


% 「とりあえず、とっさの攻撃を防ぎたいだけだから、そのミニシールドでいいよ。予備で2個くらい」


% 「了解です」\\


% 三好がさくさくポチッていく。


% 武器はなぁ。ぽよん♪ シュッ♪ バン♪ じゃダメだろうしなぁ。\\


% 「それで、武器はやはり飛び道具か?」\\


% しかしスリングショットにしろ弓にしろ、なにかの反動で与える力にステータスは乗せづらい。誰にも引けないようなものがあるのならともかく。


% 火器ならそれはなおさらだ。\\


% 魔法主体というのもひとつの手だが、MPというパラメータがある以上、完全に依存するのはかなり不安だ。


% それに、魔法無効のモンスターとか定番だしな。\\


% 「14層のムーンクラン周辺は、渓谷っぽい場所のようですし……鉄球でも投げますか?」\\


% 直接投擲かー\\


% 「兵庫にF辺精工さんって金属球の専門加工会社さんがあるんですよ」


% 「流石日本、なんでもあるな」


% 「直径0.3mm~100mmくらいの球を、いろんな素材で作ってくれるんです。8cmで2kg、6cmで850gくらいですね」


% 「じゃ、8cmと6cmを100個ずつ頼んでみるか? いろいろ試してみよう」


% 「2kgの鉄球を全力で投げたら、普通の人は肩を壊すと思いますけど」


% 「そこはステータスの力でゴリ押そう。あと、投げるんなら、やっぱり斧だろ」


% 「トマホークってやつですか?」


% 「そうそう。ちょっと重めのが良いから、ブローニングのショックンアートマホークを100本くらい」


% 「なんだかもう軍隊の発注みたいになってますね」


% 「オーノー」


% 「言うと思いました」\\


% 冷たい目で俺を非難しながら、三好は注文を確定させた。\\


% 「在庫あるみたいですから、大体明日届くと思います」


% 「了解。後はルートだな」\\


% 持ってるオーブを確かめておきたいモンスターが途中にいるなら、それも狩っておきたい。\\


% 「夢のスキルと言えば、テレポートとリザレクションですか?」


% 「後、肉体強化系も地味に便利っぽいぞ」


% 「あまり犯罪に利用されなさそうで、高額で売れそうなのは、医療に応用が利くタイプですけど」


% 「回復系か……」\\


% ヒールにキュアにディスカース。あ、最後のは違うか。\\


% 「ダンジョンから連絡が出来るツールとかもあると便利だよな」


% 「量子テレポーテーションを利用した通信手段を研究しているところがあるそうですよ」


% 「次元?が違っても有効なのかね?」


% 「量子エンタングルメントをダンジョンで確認する実験は準備されているそうです」


% 「へー。早く実用化するといいな」\\


% 一応そう言ってはみたけれど、凄そうという以外、何を言っているのかさっぱりわからん。\\


% 「その前にダンジョン素材でなんとかなりそうな気もするんですけど」


% 「そのココロは?」\\


% 三好が代々木ダンジョンの階層マップを表示して、9層をタップした。\\


% 「これです」\\


% そこに表示されたのは、コロニアルワームと呼ばれるモンスターだ。\\


% 「聞かないな」


% 「あまりにもうっとおしいので放置されているモンスターです」\\


% 次の層に降りる階段までのルートから外れた場所にいるモンスターを狩りに行くには、何かの動機付けが必要だ。


% そういうものがなかったり、支払うものに対して得るものがあまりに少なかったりするモンスターは、基本的に放置される傾向にある。


% これも放置されているモンスターのひとつらしい。\\


% 「で、それが?」


% 「コロニアルワームは、小さな群体と、大きな本体から構成されているモンスターです」\\


% 最初に接敵した自衛隊の部隊は、本体を巣だと思ったそうだ。\\


% 「群体側のワームは、積極的にいろいろなものを襲って食べますが、本体は特になにも襲ったりしないそうです」


% 「つまり、外部からのエネルギーは群体しか摂取していないってことか?」


% 「植物の根にあたるような器官も見つかっていないのでそうとしか考えられないそうです。なのに、群体の摂取したエネルギーだけで本体も成長するって、おかしくないですか?」\\


% ダンジョンのモンスターは倒すと消えてなくなってしまう。だから完全に満足な調査は行えないだろう。


% 確かにおかしいと言えばおかしいが……\\


% 「だから、群体の部分と、本体の部分が、内部で繋がってるんじゃないかと思うんです」


% 「なんというファンタジー。胃袋みたいなものを共有しているってことか」


% 「そうです。そしたら、その胃袋と群体側の器官で、同じ空間を共有できたりしませんかね?」


% 「ありえる。ありえるが……アイテムは捕まえたモンスターを解剖して取り出すってわけには行かないからなぁ」


% 「そうですね。それに……」\\


% 三好が再生した動画は、極めてグロで、ものすごくビビッた。


% 群体が通路の壁を覆うように這って移動してくるありさまは、映画スクワームのラスト付近の家の中の映像のようだ。\\


% 「げぇ……」


% 「そりゃ、誰もここに向かいませんよね」


% 「まったくだ。強力な範囲魔法でも手に入れないと、近づく気にもならない」


% 「ほんとですよ。まだ食べられたくはありません」\\


% 洒落にならん。\\


% 「単なる魔法スキルなら、結構候補が居るんですけどねぇ」\\


% 11層のレッサーサラマンドラ、17層のカマイタチ、13-14層のグレートデスマーナあたりを次々と指さしていく。\\


% 「デスマーナってなんだ?」


% 「言ってみればモグラですね。ロシアデスマンっていう大きなモグラと似ているそうですよ。太い尻尾と尖った鼻が特徴です」\\


% それは尻尾を入れなくても1mはある、鼠のようなモグラのようなモンスターだった。\\


% 「こんなのがうろうろしてるのかよ。トマホークくらいじゃ、まったく通用しないんじゃないの?」


% 「なにか物理的に刃渡りがある武器もあったほうが良いかもしれません」\\


% 刃渡りか。やっぱ剣がいるってか?\\


% 「そういえば、私、バスを出し入れしてて思ったんですけど」


% 「うん?」


% 「ものを出す時って、ある程度出す位置や向きを決められるじゃないですか」


% 「そうだな」


% 「あれって、ものすごく重く尖ったものを相手の上に出したり出来ませんかね? 素早く動くものにはダメでしょうけど」


% 「質量兵器か……それは一考の余地があるな」\\


% 「魔法だって、出した後も位置が維持できるなら、水の玉を相手の頭にあわせて出し続けるとかしたら、窒息しませんか?」


% 「モンスターって酸素呼吸してるのか?」


% 「わかりません。例のパッセージ説ですけど、深いダンジョンがもしも本当に通路で、向こうの世界と繋がっているなら、気圧の違いや空気を構成する気体のバランスの違いが大問題になりそうなものなんですけど、3年経った今でも、そんなことは起こってないですから」


% 「むこうとこっちで気圧も空気の成分も同じってことか? だから生物?も酸素呼吸だってか?」


% 「そうでなければ、やはり空間自体は断絶していて、テレポートみたいな機能で繋がってるとかでしょうか」\\


% ダンジョンの中と外では電波が届かない。各フロア同士も同様だ。


% そもそも現実に地下を占有していることはあり得ないサイズなのだ。別空間だとしか思えない。\\


% 「大体おかしいと思いませんか? 先輩」


% 「なにが?」


% 「モンスターが徘徊する異界との接点? そんな場所に多くの人が侵入してるんですよ? 未知の病原体とかいないんですかね? 検疫とかありませんよね? 防疫とかどうなってるんです?」\\


% 確かに。


% 同じ地球ですら、他国とのやりとりに検疫は実施されている。なのにダンジョンでは俺の知る限りそんなことは行われていなかった。


% 一応公開されるまでの間に、危険な細菌や生物はモンスター以外発見されなかったということだが……\\


% 「異世界の細菌は酸素で死滅する、なんてことも考えられますけど。モンスターは生きてますし」


% 「そう言う話は、俺たちがここで考えていても結論はでないな。まあ、質量兵器のアイデアはよかった。1tくらいの杭でも用意するか?」\\


% あんまり重いと、保管庫では持ちきれない。\\


% 「尖らせるなら100Kgくらいのものも使い勝手がいいかもしれません。ダンジョンの高さが無制限にあるなら、市販品で一番面白いのは5cmくらいの太さがある鉄筋ですかね」\\


% もっとも使用する空間の方に、それを運用する高さがないんですけどと、三好が苦笑しながら言った。


% 鉄筋か……そういえば、あのとき落ちた鉄筋が、この変な運命の始まりだったな。あれは一体、何に当たったんだろう。\\


% 「加速度が付けられたら便利なんだけどな」


% 「あ、それ、出来そうな気がします。後で練習してみます。しかし流石にトン級の鋳造ってことになると特注ですよね。3Dで設計図作って見積もりとってみますが、今回は間に合いませんね」\\


% 加速度って付けられそうなのかよ。


% 俺も、ピンポン球とかで練習してみるか。できるなら、STRよりINTやDEXをあげた方が良いのかな?


% その辺もあとで検討してみよう。\\


% 「ああ、よろしくな」\\


% そこで呼び鈴が鳴らされた。\\


% 「あ、来たんじゃないですか?」\\


% 画面を確認して門を開けると、表のスペースに車が入ってくる気配がした。\\


% ◇◇◇◇◇◇◇◇\\


% 納車されたキャンピングカーは、結構でかかった。\\


% 「いや、三好、凝りすぎだろ……」\\


% 室内に入って、内部を見た俺は、思わずそう呟いた。


% ベースはドリーバーデン25ftらしいが、窓が全部潰されていて光が漏れないようになっている。


% ダイネットと奧のベッドには大きなモニタがぶら下がっていて、周囲の監視カメラの映像が表示されていた。\\


% 「ダンジョン内では、騒音もよくないでしょうから、電源は全部燃料電池ですよ。超高コストです!」\\


% なんだか三好が嬉しそうだ。新しい技術は使ってて楽しいもんな。\\


% 「PEFC(*1)とDMFC(*2)の併用です。念のためにたくさん積んでおきましたけど、ファンの音がうるさそうだったので、いろいろと工夫がしてあります」\\


% 確かに大した音は聞こえない。\\


% どうせ食事は全部保管庫の中だ、本格的な料理の必要はほぼ無いからか、キッチン部分はとても簡略化されていた。\\


% 三好はいろいろと説明してくれていたが、俺はと言えば、アメリカンな内装なのになんでドリーバーデンなんだろうと下らないことを考えていた。


% オールドアメリカンとイギリス風は似たようなものだと言われればそうなのかもしれない。\\


% いずれにしてもこれでおおまかな準備は終了だ。


% 明日注文したものが届けば、そのままダンジョンに潜ることになるだろう。


% 初めての冒険らしい冒険に、俺はちょっとワクワクしていた。\\

