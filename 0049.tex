% \subsection{§049 さまよえる館(前編) 11/27 (tue)}
\subsection{§049 さまよえる館(前編) 11/27 (tue)}

% それはシャワーを浴びた後、シューティングゲームに夢中になっている三好を横目に見ながらソファーに寝そべって、そろそろ寝るかなと思い始めた頃に起こった。\\


% 「先輩!」\\


% 三好の焦ったような声に、飛び起きた俺は、彼女の側に駆け寄った。\\


% 「どうした?」


% 「あ、あれ」\\


% 三好が指さす先のモニタには、どう考えても今までそこにあったとは思えないものが映っていた。\\


% 「……洋館?」\\


% 丘の下。たしかにさっきまでは墓場だったはずの場所に、中世の貴族の館のような洋館がたたずみ。周辺からアンデッドが消えていた。\\


% 「なんだ、あれ? 三好、何かしたのか?」\\


% 三好はフルフルと首を振ると、単に敵を倒してただけですけど、と言って原因の検証を始めたようだった。\\


% 「時間制か、何か特殊なモンスターを倒したか、そうでなければモンスターを倒した数だとか」\\


% 後は俺たち自身が別の場所に飛ばされたという可能性もあるが……\\


% 「周囲の地形は、あれが出るまでと完全に一致してますから、その可能性は薄いですね」


% 「じゃあ、幻覚とか」


% 「マップ作成用の超音波センサーにも反応してますよ」\\


% つまり、突然現れたあれは、物理的にそこにあるわけだ。\\


% 「うーん。出現時間や月齢にも何か特殊な値はないですし、登場する最後に倒したのはゾンビみたいですけど、特に特殊な個体って感じはしないですね」\\


% 録画された監視カメラの映像を巻き戻しながら三好が言った。\\


% 「数だというなら……今日私が10層で倒したゾンビの数が373体目に出現したってことくらいでしょうか」\\


% 373体目? ていうか、そんなに倒してたのかよ! そんなにいたこともにも驚くけどな。\\


% 「それって、なにか特別な数字なのか? 666とかみたいに」


% 「うーん……373は回文素数ですね」


% 「なにそれ?」


% 「前から呼んでも後ろから呼んでも同じ数になる素数です」


% 「だけど、そんな数いっぱいあるだろ?」\\


% 11だって101だって131だってそうだ。\\


% 「373は、小さい方から数えて1(・)3(・)番目の回文素数ですね。先輩の言う地球の文化的っぽくないですか?」


% 「……んじゃ、さしずめここはゴルゴタの丘か?」\\


% 「確かにスケルトンは一杯いました」と言って、三好は笑った。\\


% ゴルゴタはアラム語由来のギリシャ語で「頭蓋骨」だ。


% ゴルゴタの丘に、Gランクの俺。うーん、俺のアーマライトが火を噴くぜ……って、この言い回しは下品すぎる。(*1)\\


% 「墓場の洋館と来れば、相手はヴァンパイアの一族なんてのが定番だが……」\\


% しかしそんなモンスターはいままで見つかっていない。ウェアウルフはいるらしいが、人の姿にはならないようだ。


% この世界に犬神明(*2)は存在しないのだ。まだ。\\


% 「で、どうします? あれがいつまであそこにあるのかもわかりませんけど」\\


% ゾンビを1日に373体倒すと現れる(かもしれない)館、ね。


% しかもお誘いを受けたかのように、周囲からモンスターが消えてしまうとか……\\


% 「十字架はおろか、銀の玉も聖水もニンニクすらもないけれど、ここまで来たら行ってみるしかないか?」


% 「ですよね!」\\


% ◇◇◇◇◇◇◇◇\\


% 俺達は入念に準備をした後、拠点車を出て、それを仕舞った。戻ってこれるかどうか分からないからだ。


% あたりにアンデッドの姿はなかった。\\


% 丘を降りて館に向かうと、少し錆びた風合いの、複雑な花と蔦のようなモチーフが刻まれた、両開きの鉄の門が俺たちを出迎えた。


% 門柱には奇妙な文字のようなものが描かれている。\\


% 「楔形文字……っぽいけど、違うな。象形文字、とも言えないか……」


% 「ソラホト(*3)に出てきそうな文字ですね」


% 「なにそれ?」


% 「現代の高校生がタイムスリップしてヒッタイト王国のタワナアンナになるお話です」


% 「へー。ヒッタイトあたりの文字ってことか? あ、下にアラビア数字も書かれてら」


% 「滅茶苦茶な組み合わせですね」\\


% 門柱の文字の下に小さく、1000000000000066600000000000001 と書かれている。なんだこれ?\\


% 「また素数ですね」


% 「え? これ、素数なの?」


% 「クリフォード・ピックオーバーって人が、ベルフェゴール素数と名前を付けた一部では有名な素数です。666を13個のゼロが挟んでいる、回文素数ですよ」


% 「ベルフェゴールか……」\\


% ベルフェゴールはデモノロジーじゃ地獄の7人の王子のひとりだ。人が何かを発見するのを助けてくれるという。\\


% 「ここに何か重要なものがあるってことのアナロジー?」


% 「わかりません。全部がフレーバーテキストみたいに、ただの雰囲気なのかもしれませんけど。良くできてることだけは確かですね」\\


% 割り切れない世界に素数。


% 何かがありそうな場所にベルフェゴール。


% そして666と13のオンパレード。\\


% 「三好がオーブのパッケージに刻んだ魔法陣みたいなものだとすると、ダンジョンメーカーは、宗教学にも数学にも造詣が深そうじゃないか」\\


% 動画にも記録されているはずだが、念のために、スマホでも撮影しておいた。


% 門に手をかけて軽く押しただけで、微かにこすれるような高い音を立てて、それは開いた。自ら訪問者を招き入れるように。\\


% 「こういうとき、キーって音がするのはお約束なんですかね?」


% 「不気味な静けさと、薄い霧もパッケージしてな」\\


% 気分はヘルハウスのオープニングだ。\\


% 広い前庭の先にあるのは、尖塔を伴った二階建ての大きな洋館だ。それは、圧倒的に非現実的な存在感でそこに建っていた。\\


% 見上げただけでわかる。これはあれだ。


% 正気を失い、闇を内に抱きながらひっそりと「立って」いるそれだ。"and whatever walked there, walked alone." だ。(*4)\\
不管那里走过了什么  都是形单影只

% 屋敷の扉を開き中を覗いた瞬間、ガコーンなんて、キューブリック版シャイニングの効果音が聞こえてくるに違いない。\\


% 「先輩、これ……」


% 「ああ」\\


% 引き返すべきだ。


% 内なる俺はそう叫んでいた。\\


% チャーチにベラスコが棲んでいたり(*5)、糧にされた後、写真になって暖炉の上に飾られたりする、絶対にそういう類だ。


% ただなぁ……\\


% 「スーダラ節は正義ってことだな」\\


% そう呟くと俺は前庭に足を踏み入れた。わかっちゃいるけどやめられないのだ。\\


% 「……だと思いました」\\


% 三好は、あきらめたようにため息をついて、後に続いた。\\


% その瞬間、尖塔の上で大きな黒い鳥が翼を広げて、鋭い声を上げた。


% 2階の角毎に陣取っている、翼を持ったクロヒョウのような像が生を得て、一斉にこちらを振り返る。\\


% 「あれがガーゴイルで、地球ナイズされているとしたら、屋敷へ入ろうとするものを攻撃してくるとかか? 尖塔の上は、大鴉?」


% 「そのうちきっと、"Nevermore!" って啼きますよ(*6)。って、先輩、それよりあの軒先なんですけど……」\\


% 三好が気味悪そうにそう告げる。軒先?\\


% 2階の軒先には、多数の丸いものが蠢いていた。注意して見ると、それは目だった。それがクロヒョウの顔と同様、全てがこちらを注視するように視線を向けているのだ。\\


% 「げっ、気持ちわりぃ……けど、もしかしてあれがモノアイか?」


% 「ふよふよと単体で飛んでるんじゃありませんでしたっけ? あれは何というか、群体っぽいですよ。ねとっとした感じですし」


% 「気をつけろよ。あれが襲ってきたら逃げるぞ」


% 「どこへです?」\\


% どこへ? 屋敷の中は未知数だし、ローズレッドよろしく喰われるのは勘弁だ。かといって門の外へ逃げても追いかけてこられたら、逃げ切れるかどうかわからない。\\


% 「そういや、逃げる場所がないな」


% 「先輩……」\\


% 三好が残念な子を見る目つきで睨んでくる。\\


% 「あー、とりあえず外だな。門の外」\\


% 逃げながら攻撃してれば、いずれは逃げ切れる……といいなぁ。\\


% 「……了解」\\


% 屋根の上から降り注ぐ数多の視線にびくつきながら、俺たちはついに屋敷の玄関まで数mの位置へと近づいた。\\


% 「なあ、三好」


% 「はい?」


% 「中世に自動ドアってあったのかな?」


% 「古くは、紀元前のエジプトでヘロンが作ったって話がありますよ」


% 「そうか」\\


% そこでは屋敷正面の両開きの扉が、音もなく大きく開いて、俺たちの入場を待っていた。


% いや、やっぱりこれは引き返したほうが……と門の方を見ると、いつの間にか尖塔にいたはずの大きな黒い鳥が門柱に舞い降りて、羽根繕いをしていた。


% 大きな白目のない漆黒の眼球に、球状にゆがんだ世界が映り込んでいるのが見える気がした。


% そのとき、大きな声でその鳥が啼いた。\\


% "Nevermore!"\\


% 「だ、そうだ。このチャンスを逃すなってことかな?」


% 「先輩。私、引き返したら襲われそうな……気がするんですが」


% 「奇遇だな、俺もそう思う」\\


% しかしいつまでもここで緊張していても始まらない。俺達は頭上に気を配りながら、その屋敷へと足を踏み入れた。\\

