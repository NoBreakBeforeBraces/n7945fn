% \subsection{§049 さまよえる館(前編) 11/27 (tue)}
\subsection{§049 彷徨之馆(前篇) 11/27 (二)}

% それはシャワーを浴びた後、シューティングゲームに夢中になっている三好を横目に見ながらソファーに寝そべって、そろそろ寝るかなと思い始めた頃に起こった。\\
正在我洗完了澡,靠在沙发上无奈地看着依旧沉迷于FPS的三好,想着差不多该去睡了的时候,突然发生了些异样。\\

% 「先輩!」\\
「前辈!」\\

% 三好の焦ったような声に、飛び起きた俺は、彼女の側に駆け寄った。\\
听见三好口气急迫地叫我,我马上起身来到三好旁边。\\

% 「どうした?」
「怎么了?」

% 「あ、あれ」\\
「看、看那个」\\

% 三好が指さす先のモニタには、どう考えても今までそこにあったとは思えないものが映っていた。\\
在三好指着的那个显示器上,显示着怎么看也不像是刚刚有出现在那边的东西。\\

% 「……洋館?」\\
「……洋风别墅?」\\

% 丘の下。たしかにさっきまでは墓場だったはずの場所に、中世の貴族の館のような洋館がたたずみ。周辺からアンデッドが消えていた。\\
在山丘脚下,记得刚刚明明还是墓地的地方,现在却矗立着一栋如同中世纪贵族宅邸般的西式别墅。周围的亡灵们也不见了踪影。\\

% 「なんだ、あれ? 三好、何かしたのか?」\\
「那是什么啊?三好你干了啥?」\\

% 三好はフルフルと首を振ると、単に敵を倒してただけですけど、と言って原因の検証を始めたようだった。\\
三好呼呼地摇了摇头,表示自己除了打怪没有干别的事情,然后似乎开始自己默默地分析起了原因。\\

% 「時間制か、何か特殊なモンスターを倒したか、そうでなければモンスターを倒した数だとか」\\
「要么是时间制,要么是击败了什么特殊的怪物,再不然就是跟怪物的击杀数有关」\\

% 後は俺たち自身が別の場所に飛ばされたという可能性もあるが……\\
也有可能是我们自己被传送到了别的地方呢……\\

% 「周囲の地形は、あれが出るまでと完全に一致してますから、その可能性は薄いですね」
「周围的地形在那东西出现前后完全没有变化,所以那应该不大可能」

% 「じゃあ、幻覚とか」
「那,难不成是幻觉」

% 「マップ作成用の超音波センサーにも反応してますよ」\\
「可生成地图用的超声波传感器也探测到了那东西哦」\\

% つまり、突然現れたあれは、物理的にそこにあるわけだ。\\
也就是说,那东西是突然地,物理地出现在了那个地方。\\

% 「うーん。出現時間や月齢にも何か特殊な値はないですし、登場する最後に倒したのはゾンビみたいですけど、特に特殊な個体って感じはしないですね」\\
「嗯。出现时间和月相都不是什么特殊的值,出现前我打倒的最后一个也不过是个僵尸,而且那看起来也不像是什么特殊个体呢」\\

% 録画された監視カメラの映像を巻き戻しながら三好が言った。\\
三好一边倒放着记录下的监控视频,一边说道。\\

% 「数だというなら……今日私が10層で倒したゾンビの数が373体目に出現したってことくらいでしょうか」\\
「要说数字的话……那东西是我今天在第十层打倒的僵尸数达到373只的时候出现的来着」\\

% 373体目? ていうか、そんなに倒してたのかよ! そんなにいたこともにも驚くけどな。\\
373只?不是我说,你这打的也太多了吧!另外居然这么多怪啊,这也有点令人意外呢。\\

% 「それって、なにか特別な数字なのか? 666とかみたいに」
「所以,这是什么特别的数字嘛?像是666那样的」

% 「うーん……373は回文素数ですね」
「嗯……373是第13个回文素数呢」

% 「なにそれ?」
「什么鬼东西?」

% 「前から呼んでも後ろから呼んでも同じ数になる素数です」
「就是指从后往前和从前往后读都一样的素数」

% 「だけど、そんな数いっぱいあるだろ?」\\
「但这种数应该不少吧?」\\

% 11だって101だって131だってそうだ。\\
像是11啦101啦131啦不都是么。\\

% 「373は、小さい方から数えて1(・)3(・)番目の回文素数ですね。先輩の言う地球の文化的っぽくないですか?」
「373可是从小到大第\textbf{13}个回文素数呢。是不是有点前辈说的地球文化的感觉了?」

% 「……んじゃ、さしずめここはゴルゴタの丘か?」\\
「……那,难不成这里是\ruby{各各他}{Γολγοθᾶς}山?」\\

% 「確かにスケルトンは一杯いました」と言って、三好は笑った。\\
「单论骷髅的话这里倒确实是大把呢」三好笑着回答。\\

% ゴルゴタはアラム語由来のギリシャ語で「頭蓋骨」だ。
各各他是译自阿拉姆语的希腊语词汇,意为「髑髅」。

% ゴルゴタの丘に、Gランクの俺。うーん、俺のアーマライトが火を噴くぜ……って、この言い回しは下品すぎる。(*1)\\
各各他山上,G等级的我。嗯,尝尝老子M16的威力吧……不行,这话太下流\footnote{原文为俺のアーマライトが火を噴くぜ。这一句式也有暗喻性行为的用法。}了。\ofnote{骷髅13(ゴルゴ13) / 斋藤隆夫\\过于有名的长寿作品。不多解释了。再要解释那就「…………」}\\

% 「墓場の洋館と来れば、相手はヴァンパイアの一族なんてのが定番だが……」\\
「说到墓地和西式别墅,那按惯例敌人就确定是吸血鬼一族了呢……」\\

% しかしそんなモンスターはいままで見つかっていない。ウェアウルフはいるらしいが、人の姿にはならないようだ。
可时至今日,却也依旧没有发现过那种怪物。虽说狼人倒是发现了,但好像是不能变成人型的。

% この世界に犬神明(*2)は存在しないのだ。まだ。\\
看来这个世界不存在犬神明\ofnote{人狼小子系列(ウルフガイ・シリーズ) / 平井和正\\与死灵猎人系列并称双壁的,在出版幻魔大战系列前作者的代表作。\\作者之前的作品也挺有意思的。比如说阿雪\footnote{指平井和正作品《アンドロイドお雪》(机器人阿雪)}什么的。}这么个人。暂时来说。\\

% 「で、どうします? あれがいつまであそこにあるのかもわかりませんけど」\\
「所以,怎么办?也不知道那东西是不是一直都在的」\\

% ゾンビを1日に373体倒すと現れる(かもしれない)館、ね。
一天内打倒373只僵尸(大概)就会出现的别墅,么。

% しかもお誘いを受けたかのように、周囲からモンスターが消えてしまうとか……\\
而且四周的怪物也如同受到了吸引般,全部消失了……\\

% 「十字架はおろか、銀の玉も聖水もニンニクすらもないけれど、ここまで来たら行ってみるしかないか?」
「虽然我们确实两手空空,别说十字架,就连银珠圣水,甚至连大蒜都没带,不过既然来都来了,还是去一趟看看咯?」

% 「ですよね!」\\
「好嘞!」\\

% ◇◇◇◇◇◇◇◇\\
\sqsplit\\

% 俺達は入念に準備をした後、拠点車を出て、それを仕舞った。戻ってこれるかどうか分からないからだ。
我俩仔细地准备完毕,走出了据点车,然后把车子收了起来。毕竟说不定我们可能不回来这了。

% あたりにアンデッドの姿はなかった。\\
四周不见一只亡灵。\\

% 丘を降りて館に向かうと、少し錆びた風合いの、複雑な花と蔦のようなモチーフが刻まれた、両開きの鉄の門が俺たちを出迎えた。
我们朝着别墅,走下小山,展现在我们面前的便是一对略带锈迹的铁门,而铁门上则镌刻着一些复杂的花和爬山虎纹样。

% 門柱には奇妙な文字のようなものが描かれている。\\
门柱上,则描画着一些奇特的文字一样的东西,\\

% 「楔形文字……っぽいけど、違うな。象形文字、とも言えないか……」
「楔形文字……有那么点意思,但应该不是。象形文字,也不大对头……」

% 「ソラホト(*3)に出てきそうな文字ですね」
「感觉像是天岸\ofnote{天是红河岸(天は赤い河のほとり) / 篠原千绘\\天读作「そら」对单行本用户来说实在是不便。看单行本封面标题上既没有注ruby也没有附罗马字。\\直到卷末一不小心就会看漏的Flower Comics宣传页里才第一次出现天是读「そら」的提示。\\这是我遇见的,玻璃假面之后的第二部一读就停不下来的少女漫画系作品。}里会出现的文字呢」

% 「なにそれ?」
「那什么书啊?」

% 「現代の高校生がタイムスリップしてヒッタイト王国のタワナアンナになるお話です」
「现代的高中生穿越时空到了赫梯王国成为塔瓦纳安娜的故事」

% 「へー。ヒッタイトあたりの文字ってことか? あ、下にアラビア数字も書かれてら」
「哦~。你是说像赫梯那一片的文字嘛?啊,下面还有阿拉伯数字」

% 「滅茶苦茶な組み合わせですね」\\
「这密密麻麻的是啥啊」\\

% 門柱の文字の下に小さく、1000000000000066600000000000001 と書かれている。なんだこれ?\\
在门柱的文字下,小小地刻着1000000000000066600000000000001 这么一串数字。这啥啊?\\

% 「また素数ですね」
「又是个素数呢」

% 「え? これ、素数なの?」
「诶?这玩意,是素数?」

% 「クリフォード・ピックオーバーって人が、ベルフェゴール素数と名前を付けた一部では有名な素数です。666を13個のゼロが挟んでいる、回文素数ですよ」
「一个叫\ruby{柯利弗德}{Clifford}·\ruby{皮寇弗}{Pickover}的人,将其称为\ruby{贝尔芬格}{Belphegor}素数,在某些圈子里很有名的一个素数。666两边各被13个0夹住,所以是回文素数哦」

% 「ベルフェゴールか……」\\
「贝尔芬格啊……」\\

% ベルフェゴールはデモノロジーじゃ地獄の7人の王子のひとりだ。人が何かを発見するのを助けてくれるという。\\
贝尔芬格在恶魔学里是七宗罪之一。说是一个会帮助人们发现事物的恶魔。\\

% 「ここに何か重要なものがあるってことのアナロジー?」
「难不成这是在暗示有什么重要的东西么?」

% 「わかりません。全部がフレーバーテキストみたいに、ただの雰囲気なのかもしれませんけど。良くできてることだけは確かですね」\\
「不清楚。也有可能全部都只是渲染气氛用的文字。只能说气氛确实烘托的不错呢」\\

% 割り切れない世界に素数。
不完美的世界与素数。

% 何かがありそうな場所にベルフェゴール。
在这煞有其事的地方出现的贝尔芬格。

% そして666と13のオンパレード。\\
然后又是666与13的轮番上场。\\

% 「三好がオーブのパッケージに刻んだ魔法陣みたいなものだとすると、ダンジョンメーカーは、宗教学にも数学にも造詣が深そうじゃないか」\\
「要是这跟三好你在宝珠的盒子上刻的魔法阵一样的东西是一回事的话,那这个迷宫的创造者,对宗教学跟数学看来是造诣颇深嘛」\\

% 動画にも記録されているはずだが、念のために、スマホでも撮影しておいた。
虽然已经有在录视频,但以防万一我还是掏出手机再拍了张照。

% 門に手をかけて軽く押しただけで、微かにこすれるような高い音を立てて、それは開いた。自ら訪問者を招き入れるように。\\
手碰上门,轻轻一推,门便发出转轴摩擦时的尖锐响声,开了开来。如同主动招徕访客一般。\\

% 「こういうとき、キーって音がするのはお約束なんですかね?」
「这种时候,吱嘎~的声音简直成了惯例了呢?」

% 「不気味な静けさと、薄い霧もパッケージしてな」\\
「然后再配上令人不安的死寂和薄薄的迷雾」\\

% 気分はヘルハウスのオープニングだ。\\
感觉就像进鬼屋一样。\\

% 広い前庭の先にあるのは、尖塔を伴った二階建ての大きな洋館だ。それは、圧倒的に非現実的な存在感でそこに建っていた。\\
宽广的前院尽头,是一栋带着尖塔的二层西式别墅。立在前方的这屋子,散发着压倒性的非现实的存在感。\\

% 見上げただけでわかる。これはあれだ。
一抬头就明白了。这就是那个。

% 正気を失い、闇を内に抱きながらひっそりと「立って」いるそれだ。"and whatever walked there, walked alone." だ。(*4)\\
那个失去了理智,在黑暗里抱着自己静静地「站着」的那个。\trans{"and whatever walked there, walked alone."}{不管那里走过了什么,都是形单影只}那个。\ofnote{鬼入侵(The Haunting of Hill House) / \ruby{雪莉}{Shirley}·\ruby{杰克森}{Jackson}\\鬼屋恐怖题材鼻祖。斯蒂芬·金的作品撒冷镇(Salem's Lot)的开头就有这部作品的引用。\\虽然是两次被改编为电影,第二次还得了5个金酸莓奖的传说作品,但在营销上却做的十分成功。}\\

% 屋敷の扉を開き中を覗いた瞬間、ガコーンなんて、キューブリック版シャイニングの効果音が聞こえてくるに違いない。\\
正在我们打开大门,瞄了眼里面的时候,我听见了咯咚一声,绝对是库布里克的电影闪灵里的效果音的声音。\\

% 「先輩、これ……」
「前辈,这……」

% 「ああ」\\
「嗯」\\

% 引き返すべきだ。
绝对应该开溜。

% 内なる俺はそう叫んでいた。\\
我心里的声音朝我喊道。\\

% チャーチにベラスコが棲んでいたり(*5)、糧にされた後、写真になって暖炉の上に飾られたりする、絶対にそういう類だ。
教堂里住着贝拉斯科啦\ofnote{Hell House / Richard Burton Matheson\\大概电影比原作有名的一部作品。原作也兼当剧本。\\讲述的是现在看来十分可笑的\ruby{艾默里奇}{Emeric}的纠葛和执着让几个人遭受了十分离奇的事情的故事。\\艾默里\ruby{奇}{ヒ}原文是Emeric,所以发音说不定应该是埃默里\ruby{克}{ク},但听里面菲舍尔的发音感觉更应该是艾默里\ruby{奇}{キ}。不过字幕里用的是ヒ。},给了食物之后就会变成照片,装饰在暖炉上啦,肯定是些这种的。

% ただなぁ……\\
不过啊……\\

% 「スーダラ節は正義ってことだな」\\
「管他呢,一不做,二不休」\\

% そう呟くと俺は前庭に足を踏み入れた。わかっちゃいるけどやめられないのだ。\\
我一边说着,一边迈出了脚步,跨进了前院。正所谓明知山有虎,偏向虎山行。\\

% 「……だと思いました」\\
「……就知道前辈会这样」\\

% 三好は、あきらめたようにため息をついて、後に続いた。\\
三好则如同放弃了一般,叹了口气,然后跟在了我的后面。\\

% その瞬間、尖塔の上で大きな黒い鳥が翼を広げて、鋭い声を上げた。
就在这一瞬间,尖塔上一只黑色的大鸟张开双翼,发出了刺耳的叫声。

% 2階の角毎に陣取っている、翼を持ったクロヒョウのような像が生を得て、一斉にこちらを振り返る。\\
然后,二层每个角上都有的有翅膀的黑豹般的石像全部动了一下,头看向这边。\\

% 「あれがガーゴイルで、地球ナイズされているとしたら、屋敷へ入ろうとするものを攻撃してくるとかか? 尖塔の上は、大鴉?」
「要那些都是“地球文化化”了的\ruby{石像鬼}{Gargoyle}的话,我们走进屋子的时候是不是会来攻击我们啊?尖塔上那个,是大乌鸦?」

% 「そのうちきっと、"Nevermore!" って啼きますよ(*6)。って、先輩、それよりあの軒先なんですけど……」\\
「肯定一会就会"Nevermore!"地叫起来啦。\ofnote{The Raven / Edgar Allan Poe\\过于有名,没啥好说的。Nevermore.}话说前辈,比起那些,你看屋檐……」\\

% 三好が気味悪そうにそう告げる。軒先?\\
三好一脸难受的表情向我说道。屋檐?\\

% 2階の軒先には、多数の丸いものが蠢いていた。注意して見ると、それは目だった。それがクロヒョウの顔と同様、全てがこちらを注視するように視線を向けているのだ。\\
在二楼的屋檐上,大量圆形的东西正在蠕动。仔细一看,那些居然都是眼睛。而且他们也都跟黑豹的脸一样,全部朝向着我们,如同在注视着这边一般。\\

% 「げっ、気持ちわりぃ……けど、もしかしてあれがモノアイか?」
「哇,好恶心……啊不过,难不成这些就是独眼?」

% 「ふよふよと単体で飛んでるんじゃありませんでしたっけ? あれは何というか、群体っぽいですよ。ねとっとした感じですし」
「资料里独眼不是单独出现,呼呼飞着的东西吗?但那些东西,怎么看都是成群的啊。而且看起来还粘糊糊的」

% 「気をつけろよ。あれが襲ってきたら逃げるぞ」
「多当心啊。要是那东西攻击过来了就赶紧跑吧」

% 「どこへです?」\\
「往哪跑?」\\

% どこへ? 屋敷の中は未知数だし、ローズレッドよろしく喰われるのは勘弁だ。かといって門の外へ逃げても追いかけてこられたら、逃げ切れるかどうかわからない。\\
往哪?屋里什么情况不得而知,乖乖被攻击更是敬谢不敏。可是就算要往门外面跑,能不能跑的出去还是个未知数。\\

% 「そういや、逃げる場所がないな」
「说起来,跑也没地方跑来着」

% 「先輩……」\\
「前辈……」\\

% 三好が残念な子を見る目つきで睨んでくる。\\
三好以看笨孩子的目光看着我。\\

% 「あー、とりあえず外だな。門の外」\\
「啊,总之往外跑。门外面就是了」\\

% 逃げながら攻撃してれば、いずれは逃げ切れる……といいなぁ。\\
边跑边打的话,总归能逃出去……就好了。\\

% 「……了解」\\
「……明白了」\\

% 屋根の上から降り注ぐ数多の視線にびくつきながら、俺たちはついに屋敷の玄関まで数mの位置へと近づいた。\\
于是,我俩一边沐浴在屋顶上无数的目光之下,一边缓缓接近到离大门只有数米的位置。\\

% 「なあ、三好」
「话说啊,三好」

% 「はい?」
「嗯?」

% 「中世に自動ドアってあったのかな?」
「中世纪有没有自动门这种东西啊?」

% 「古くは、紀元前のエジプトでヘロンが作ったって話がありますよ」
「往古代说的话,据说公元前希罗就在古埃及做出来过了」

% 「そうか」\\
「这样啊」\\

% そこでは屋敷正面の両開きの扉が、音もなく大きく開いて、俺たちの入場を待っていた。
眼前双开的大门,悄无声息地打开,等待着我们的进入。

% いや、やっぱりこれは引き返したほうが……と門の方を見ると、いつの間にか尖塔にいたはずの大きな黒い鳥が門柱に舞い降りて、羽根繕いをしていた。
唉,果然还是回去比较……我想着这些,看了眼门,却发现本应在尖塔上的巨大黑鸟现在却飞到了门柱上,拨弄着自己的羽毛。

% 大きな白目のない漆黒の眼球に、球状にゆがんだ世界が映り込んでいるのが見える気がした。
我仿佛看见了,那双漆黑的眼球里映出的扭曲成球状的世界。

% そのとき、大きな声でその鳥が啼いた。\\
突然,那只鸟大声地叫了起来。\\

% "Nevermore!"\\
"Nevermore!"\\

% 「だ、そうだ。このチャンスを逃すなってことかな?」
「它,这么说道。要不我们现在溜吧?」

% 「先輩。私、引き返したら襲われそうな……気がするんですが」
「前辈。我感觉要是现在回头的话……可能会被袭击诶」

% 「奇遇だな、俺もそう思う」\\
「真是神奇呢,我也这么觉得」\\

% しかしいつまでもここで緊張していても始まらない。俺達は頭上に気を配りながら、その屋敷へと足を踏み入れた。\\
不过总在这里自己吓自己也不是个办法。于是,我俩便一边小心着头上的动向,一边走进了屋子。\\
