% \subsection{§060 オーブの移送は命がけ? 12/2 (sun)}
\subsection{§060 オーブの移送は命がけ? 12/2 (sun)}

% オーブ受け渡し当日、俺達は代々木八幡から小田急で新宿へと向かった。\\


% 「先輩。車のほうが良くないですか?」


% 「日本で大々的にテロを起こすならともかく、そうでなければ人の多い大量輸送機関のほうが安全な気がしないか?」\\


% 小田急は新宿までずっと住宅街で、大騒ぎを起こさずに襲えるような場所はない。そもそも小田急に乗るかどうかも分からないんだから、事前になにか大がかりなことを仕込む時間はなかっただろう。\\


% 「そういえば、鳴瀬さん。さっそく館の情報を公開したみたいですね」


% 「そうだな。ちゃんと俺達が出ているところはカットしてくれてたじゃないか」


% 「あれは私が編集したんですー」


% 「あ、そうなの?」\\


% 道理できれいに俺達が出ているところがカットされていたはずだ。


% そのまま出すと、編集段階で身バレするしな。\\


% 「ボランティアですよ。えっへん」


% 「お疲れ様です」\\


% JDAのお迎えは断ったままだ。お迎えが入れ替わるなんて定番中の定番だし、アルスルズやステータス任せの逃亡に、警護の人間がいると邪魔だからだ。


% DADあたりは、ひょっとして何処かで見ているかも知れないが……\\


% 「新宿からはどうします? 普通なら総武線か新宿線ですけど」


% 「乗り換えに便利なのは総武線だから、そっちに乗ろう。改札を出たらすぐそこだ」


% 「ええ……そんな理由で?」


% 「外国人の多い新宿駅を、長い距離歩きたくないだろ? それに地下鉄は逃げ場がないから」\\


% 地下はヤバイ。爆破されたら生き埋めだし、前後を押さえられたら側道でもないかぎり逃げようがない。


% さすがに爆破はないと思うが、なんとなく地上の方が安心できる気がする。\\


% 総武線のホームに上がると、俺達は、後ろから4両目を目指しつつゆっくりと歩いた。\\


% 「2両ほど後ろ、3人組の外国人がいるな」


% 「新宿ですよ? そんなの大勢いますって」


% 「観光客特有のうわついた感じがないし、空き座席を確認するでもなくこっちと同じ速度で歩いてるだろ。まあ、見てろ」\\


% ことさらにゆっくりと歩いた俺達は、電車が発車する寸前に、後ろから4両目の車両に飛び込んだ。


% もちろんそのままドアが閉まって発車するのは映画の中だけの話だ。


% 新宿駅なら、ホームを監視している駅員が、その様子を見逃すはずがなかった。締まりかけたドアはもう一度開いて、再度時間をおいてから閉じた。\\


% 「後ろの人達、移動してきますかね?」


% 「新宿の乗車率を舐めちゃいけない。車内移動とかほぼ無理だろ」\\


% ギュウギュウとは言わないが、それなりに混んでいて、車内を歩いて移動するのは無理そうだ。\\


% 「それにこの電車、きっと四谷の手前で事故が起こるぜ?」


% 「どうしてわかるんです?」\\


% 昨日ストリートビューで今日のルートを確認してみた。


% 車内の誰かを襲って持ち物を奪うという観点で、だ。\\


% 総武線の新宿-市ヶ谷間には、両側が木々に覆われ、まわりから電車が視認しにくくなる箇所が1カ所だけ存在する。\\


% 「襲ってくるなら、外濠公園だからな」\\


% 電車はゆっくりとスタートした。\\


% 「サイモンさん達って、もしかして今日もついてきてるんでしょうか?」


% 「DADの関係者が何処かにいてもおかしくはないが……サイモン達は受け取り側の護衛に行ってるんじゃないか?」\\


% まあ、不確かなものをアテにするのはよろしくない。\\


% 電車が首都高4号新宿線と併走し始めた頃、車内アナウンスが千駄ヶ谷を告げた。\\


% 「次は~千駄ヶ谷~、千駄ヶ谷~」\\


% 「三好、降りる準備」


% 「え?」\\


% 車内アナウンスのコールが終わり、電車の速度が少しずつ遅くなっていく。\\


% 「いいか、階段を下りて、ダッシュで改札を抜けたら、右へ曲がれ。そしたら、すぐエクセルシオールカフェがあるから、その先を右に曲がって、ガードをくぐるんだ」


% 「だ、ダッシュですか? 自慢じゃないですが、体力にはあまり自信が……」


% 「まさかヘルハウンドにまたがるわけにはいかないだろ?」


% 「そりゃまあ、そうなんですけど」\\


% 停車してドアが開いた瞬間、俺達は飛び出して、目の前にある下り階段を駆け下りた。


% PASMOをかざして、改札を駆け抜けた瞬間、三好が音を上げた。\\


% 「せ、先輩! もうムリ!」\\


% それを聞いた俺は、ステータスに任せて、三好をひょいと荷物のように小脇に抱えると、エクセルシオールカフェの前を駆け抜けた。


% 何でもかんでも写真に撮る輩が、硝子の向こうから撮影してなきゃいいんだけどな。へたすりゃ誘拐の現行犯に見えかねない。\\


% 「ひ、ヒド! ここはお姫様だっこじゃないんですか!?」


% 「やかましい、舌を噛むから、黙ってろ!」


% 「んぐぐっ」\\


% すぐに右折すると、中央緩行線の八幡前ガードだ。


% 幸いガード下に人気(ひとけ)はない。ここなら撮影される恐れもなさそうだ。\\


% 人混みの中での生命探知スキルはマーキングでもしていない限り意味はない。このスキルは種の区別は出来ても、敵と味方を区別しないからだ。


% だから、このスキルが仕事をする場所に向かおうと最初から考えていた。\\


% 中央緩行線の八幡前ガードの出口の右側は、新宿御苑だ。


% ただし、その場所は、3~4mは高さがある石垣の上に、2m近い鉄柵が張り巡らされている。\\


% 俺は三好を抱えたまま、石垣を駆け上がった。\\


% ここの石垣は、しばらく行くと整然と積み上げられたものになるが、ガードを出てすぐのところは、自然石風の凸凹が顕著なのだ。苔で滑りさえしなければ、今のステータスで駆け上がるくらい、わけはない。


% そしてそのまま、高さ制限バーと呼ばれる黄色に黒の縞々が書かれた鉄骨へ足をかけると、一気に2m近い鉄柵を跳び越して御苑の森の中に飛び込んだ。\\


% 「せ、先輩、どっかのアクションスターですか?!」


% 「ステータス様々ってやつだな。いくら訓練された軍人でも、おいそれとはついて来れないはずだ」\\


% 千駄ヶ谷駅は、新宿御苑に隣接しているが、現在そちら側へ出る手段はない。


% つまり、追いかけてこようとした場合、改札を出た後、俺達が来た反対方向へ走るか、俺達を追いかけて、このずっと先の坂の上で2mの鉄柵を跳び越えるか、さらに先の千駄ヶ谷門から正規に入るしかないはずだ。\\


% 千駄ヶ谷門は遠すぎる。通常は2番目の選択肢しかないだろうが、飛び越えた場所は御苑内の桜園地の端の森だ。観光客が入れるような場所ではない。


% ここに入ってくるやつがいれば、そいつが俺達にとっての危険人物だ。\\


% 「確かにすごいんですけど、御苑って入場料がいったような……」


% 「うぐっ……」\\


% 三好が冷静に突っ込んできた。


% 確かにいる。200円(*1)だ。緊急避難ということで、許して貰おう。\\


% 「それに、出るときもチケットがいりますよ?」


% 「マジで?!」


% 「前言撤回します。スターはスターでも、ントマン付きですね」\\


% スタントマンかよ! しかし超一流の裏方は格好いい気がするな。\\


% 「誰が超一流って言ったんですか」\\


% 三好がテレながらふくれてる。貴様ツンデレだな?\\


% しかたない、大木戸門の左側の柵を跳び越えるしかないか。駐車場側は警備員が一杯だけど、素早く飛べばごまかせないかな?


% そんなことを考えつつ、下の池に出た。そこには、カメラを抱えている人の群れが……なんじゃこりゃ。俺は思わず、三好を下ろした。\\


% 「下の池って、楓の名所ですよ。今はベストシーズンですね」\\


% くそっ、なんてこったい。


% 確かに美しく色づいた大きな楓が、水面にその姿を映している。通路には三脚が林立していて、カメラを持っていない人を探す方が早かった。\\


% そのとき、桜園地の向こう側、人の入らないエリアに反応があった。\\


% 「やっぱり、誰かが追いかけてきてるぞ」


% 「いやー、スリル、ありますねー」


% 「のんきなヤツだな。番犬連中はちゃんとガードに付いてるのか?」


% 「先輩の影にドゥルトウィンが。私の影に、カヴァスとアイスレムが。グレイシックは事務所でお留守番です」\\


% 俺の影にもいるのかよ。頼むぞ犬っころども。嗜好品の餌分は働け。


% 狙撃も防げると豪語してたからな。そう言うレベルの襲撃になったら、そこに期待するしかない。\\


% バラ花壇の横を通って、大木戸門方面へ抜けると、左手に大温室が見えてくる。\\


% 「どうやら、洋らん展をやってるみたいですね」\\


% 三好が指さした大温室には、「第30回 新宿御苑洋らん展」の表示があった。どうやら最終日だ。


% 温室の人混みはこのせいか! なんと間の悪い。まあ、裏手なら誰も見てないだろう(希望的観測)。硝子越しに目撃されそうな気もするが。\\


% まったく全員が映像記録の手段を持ち歩いているなんて、なんとも困った時代だな。\\


% 「よし、仕方ない」


% 「待って下さい!」\\


% 三好が強行突破しようとする俺を止めると、てくてくと大木戸門へと歩いていった。\\


% 「すみませーん。私たち、チケット間違って捨てちゃったみたいで、なくしちゃったんですけど、出ても良いですか?」\\


% と、ニコニコしながらそう言った。\\


% 「ん? 仕方ないな。次から気をつけてね」


% 「はい、ありがとうございます。楓がとてもきれいでした、また来ますね」


% 「はい、ありがとう」\\


% 好々爺然と崩れた笑みで、受付のお爺さんが俺達を見送ってくれた。


% なんというコミュ力。\\


% 「凄いな、お前」


% 「入る時ならともかく、出る時なんてあんなもんですよ。こんなに警備員だらけの場所で強行突破なんてアホですか、先輩は」


% 「ぐぬぬ」


% 「後ろの人達はどうするんでしょうね?」\\


% 三好が笑いながら、そういった。


% そういや、ついてこないな?\\


% ◇◇◇◇◇◇◇◇\\


% 「見失った?」\\


% なんと勘の良い連中だ。


% 車両越しにちらりと我々の姿を確認すると、すぐに千駄ヶ谷で電車を降りて、そのまま新宿御苑へ逃げ込まれたらしいが、どうやって、御苑へ入ったのかすらわからないだと?


% ガードを出たところで、突然御苑に駆け上がったことは確かだが、階段があるのかと駆け寄っても、そこには崖しかなかったらしい。\\


% 何かのスキルを持っているという情報はなかったが、あれだけのオーブを売りに出しているのだ、自分で使っていてもおかしくはないか。\\


% 御苑内でも、どこかのチャチャが入ったらしい。


% 本格的な争いにはならなかったようだが……うちと同じようなチームを派遣している国があっても不思議はない。\\


% さて、御苑の北側に抜けたとすれば、車か、丸ノ内線だ。


% この連中なら、さらにそのまま北へ抜けて新宿線を利用する可能性や、もう一度千駄ヶ谷駅へ戻って、次の電車に乗り込む可能性もあるか……\\


% 外濠公園に配置していた連中は、半分を四谷駅へ、半分を市ヶ谷方面へ向かわせた。


% 紛争地帯ならもっと派手にやってもいいんだが、ここでは縛りが多すぎる。せいぜいが事故に見せかけたアタックくらいだ。\\


% ルートの可能性が多すぎる。もはや最悪の準備をするべきか……


% 情報によると、どちらかを手に掛けさえすれば、オーブの存在は保証されないという。\\


% 男は仕方ないというように肩をすくめると、ドミール五番町ビルと、グランドヒル市ヶ谷に特殊なメンバーを派遣した。(*2)\\


% ◇◇◇◇◇◇◇◇\\


% 「それでこれからどうするんですか、先輩。丸ノ内線で四谷に出るか、そのまま北へ向かって、新宿線?」


% 「新宿一丁目交差点を渡って、新宿通りか、外苑西通りで流してるタクシーを拾おう」\\


% 予定された場所なら仕込みもあり得るが、こんだけ意表を突いた後の流しなら、そうそう仕込みは行えないだろう。\\


% 「わかりました」\\


% そういって、交差点を駆けていった三好が、速攻でタクシーを捕まえていた。\\


% いわゆるドアがスライドするタイプの、JPNTAXIだ。


% スライドドアの開閉が遅く交通量が多い通りだとプレッシャーがかかるとか、窓が開かないとか色々言われているあれだ。


% しかし広い空間は、なかなか快適だった。\\


% 「靖国通りへ出て、JDA市ヶ谷本部までお願いします」


% 「承知しました」\\


% 行き先を告げると、車は滑るように動き始め、何事もなく外苑西通りを北に進んで、富久町西の交差点で靖国通りに入った。\\


% 「一段落ですかね?」


% 「そうだな。靖国通りに入ってしまえば、さすがに大きなアクションを起こしたりはしないと思うけど……」\\


% 俺はふと不安になった。\\


% 「なんです?」


% 「いや、どのルートを通ったとしても、結局俺達の目的地は同じだろ?」


% 「そうですね」


% 「なら、もしも振り切られたとしたら、後はJDAの周辺で待ちかまえるんじゃないか?」


% 「ええ? 防衛省の目と鼻の先ですよ?」\\


% 防衛省の正門を通り過ぎ、外堀通りへ300mの標識を越えた時、緩やかに左へと曲がる道路の中央よりの対向車線を、大きなトレーラーがスピードを上げて走ってくるのが見えた。


% なんだか嫌な予感がした俺は、三好に目線で合図を送った。\\


% その瞬間、まるでタイヤがパンクしてハンドルを取られたかのように、急に右によれたトレーラーは、あっという間に横転して、コンテナ部分がこちらに向かって滑り出した。


% 遠心力で振り回されたそれは、東へと向かう3車線を完全に塞いで突っ込んできた。\\


% 右は対向車、左は境界ブロックと柵で逃げ道はない。


% 運転手はパニックになってブレーキを踏もうとした。\\


% 「止まるな! 直進しろ!」\\


% そう叫んで、ドゥルトウィンに影からアクセルを踏ませると、タクシーは急に速度を上げて、コンテナに突っ込んで行く。\\


% 「ええ?! うわあ!」\\


% 突然のことに運転手が叫び声を上げて目を瞑った。


% 横滑りしてくるコンテナにぶつかる寸前、コンテナの影から何かが飛び出した。カヴァスだ。\\


% コンテナは、カヴァスに乗り上げるように、はじかれて、丁度タクシー一台が通過する短い間だけ、くるくると宙を舞い、その直後に落下した。


% 映画なら確実にトリプルアクションでスローになるシーンだ。\\


% グワシャーン!と凄い音を響かせたコンテナは、ゴロゴロと転がりながら都バスの防衛省前停留所を薙ぎ倒すと、そのまま道路沿いに滑って、防衛省の正門付近で止まったようだった。\\


% 「た、助かったんですか?」\\


% 運転手が呆然とした顔でそう言った。\\


% (ナイスだ、三好)


% (後で、魔結晶ですね)


% (うっ。了解)\\


% 呆然とする運転手を尻目に、そのまま惰性とクリープ現象で進んでいた車は、ホテルグランドヒル市ヶ谷の直前で停止した。


% 本来ここは横断禁止だが、後ろの大事故で、どうせ後続車はいない。\\


% 「お釣りは結構です」\\


% そういって諭吉様を置いた俺達は、急いで車を降りて道路を渡った。JDAは文字通り目の前だ。\\


% 「先輩、今の事故って……」\\


% 足早に入り口に向かって歩きながら、後ろを振り返りつつ、流石に少し青ざめた顔で三好が言った。\\


% 「こないだJDAで話しただろ?」


% 「なんでしたっけ?」


% 「ほら、俺か三好のどちらかが欠けたら預かっていたものは保証できないってやつ」


% 「ああ、確かにそんなハッタリをカマしてましたね……まさか」


% 「じゃないかと思うわけよ」\\


% あの情報が漏れていたとしたら、最悪奪えないなら、俺達のどちらかを消してしまえば闇に葬れる的な発想に到ってもおかしくはない。\\


% 「先輩、ここって、日本ですよね?」


% 「まあ、トレーラーがパンク?したのは偶然かも知れないし、全部俺達の妄想かも知れないからな」\\


% もっとも、こんなにタイミング良く靖国通りをトレーラーが走ってくるのもどうかと思うけれど。一体どこにものを運ぼうってんだよ。


% まっすぐ行けば新宿のど真ん中で、そのまま進むと、八王子や大月だぞ。コンテナなら、川崎や横浜方面だろ。\\


% パトカーや救急車のサイレンの音が聞こえてくる。


% 流石は日本の救急車両、初動が早いと感心しながら、JDAの入り口方向へ曲がろうとした瞬間、小さなターンという音と共に、三好の頭の横に5cmくらいの黒い円が生まれたかと思うと、すぐに消えた。\\


% 「アイスレム!」\\


% その瞬間、三好がワンコに指示をする。\\


% 「って、今の……」\\


% 狙撃されたのか? その弾をカヴァスあたりがカットしたのか?


% 本来ならすぐにでも遮蔽物のところまで駆けるべきなのだが、そんな訓練などされていない俺達は、思わず道路を渡ったところにあるホテルの屋上に目をやった。\\


% とはいえ、もしも影を渡った先に、狙撃した誰かがいるのだとしたら、今頃はきっと事務所に侵入しようとした連中と同じ目にあっているはずだ。\\


% 「三好、大丈夫か?」


% 「まさか本当に……アルスルズが任せとけって自信満々だったのは、伊達じゃありませんでしたね。とはいえ、狙われたって実感はまるでないんですが」\\


% 弾も見てないし、ただ音がしただけですから、と笑った。


% 確かに海外の乱射事件の映像を見ても、人が逃げ出し始めるのは、誰かが倒れてからが圧倒的に多い。


% とはいえ、ショックであることに代わりはないだろう。\\


% 俺は三好の肩を抱いて、JDAのロビーへと入っていった。\\

